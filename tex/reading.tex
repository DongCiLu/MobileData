\section{Relate Work}

\subsection {Smartphone app usage}
~\cite{xu2011identifying} comprehensively shows the aggregated spatial and temporal prevalence, locality and correlation of smartphone apps at a national scale. ~\cite{yang2015characterizing} also studied the smartphone app usage patterns. Although correlation of user mobility and app usage volume have been briefly studied in this paper, very limited results have been presented compared to our work. Also, such findings does not have much help in applications such as app launching prediction and app recommendation.

\subsection{Inferring user mobility}
There are several work in referring user mobility information from smartphone traces which build the base stone for our work.

~\cite{rose2006mobile} is the study that first revealed the great potential of using cell-phone for user mobility study. It showed several advantages of such an approach against traditional passive sensor approach as well as listed challenges of this approach.

~\cite{wang2014mobile}
Mobile phone positioning data sets contain spatio-temporal positions of millions of mobile phone users. While their application in recent human behavior studies has seen great success, it remains to be explored by the travel behavior research community. In this study, the authors discuss the potential of mobile phone positioning data as an alternative data source for travel behavior studies. The authors are particularly interested in their potential for analyses in travel behavior dynamics. Analyses in travel behavior dynamics rely on multi-day travel data. Multi-day travel diary and Global Positioning System (GPS) tracking are two primary approaches for multi-day travel data collection. Yet, both approaches are limited in several aspects. The authors discuss the relative advantages of mobile positioning over travel diary/GPS tracking as an alternative data collection technique. As an example, a real-world mobile phone positioning data set is applied to examine the variability in individuals’ activity location choices. Results demonstrate the ability of mobile phone positioning data in capturing temporal dynamics in individuals’ spatial behaviors. The authors conclude that mobile phone positioning data are a promising data source for travel behavior study. There are, however, caveats. The authors discuss potential issues with this type of data as well as their implications when used for travel behavior studies. In addition, the authors review recent studies producing novel insights into human mobility with mobile phone positioning data from other disciplines in the hope that more travel behavior researchers will join this effort

~\cite{6958169}
A novel method for separating trip periods from non-trip periods, called a “trip-separation method” hereafter, is proposed. Previous studies on such automatic “trip separation” use a fixed threshold to detect “stay periods” (i.e., periods in which a subject is staying put) and separate them from trip periods. However, this separation procedure does not always work well because the collected GPS-positioning data can fluctuate according to the surrounding environment even when the subject stays within an area. The proposed method dynamically adjusts parameters for detecting stay periods according to the surrounding environment. The experimental evaluation on the method showed precision of 81%. This is a promising result implying that sensor-based travel-behavior surveys are possible by utilizing the proposed smartphone-based method.

~\cite{6450942}
Data about people movement is nowadays easy to collect by GPS technology embedded in smartphones. GPS routes provide information about position, time and speed, but further conclusion requires either prior information or data analysis. We propose a method to detect the movement type by segmentation of the GPS route using speed, direction and their derivatives, and by applying an inference algorithm via a second order Markov model. The method is able to classify most typical moving types such as motor vehicle, bicycle, run, walk and stop.

~\cite{hoteit2014estimating}
Nowadays, the huge worldwide mobile-phone penetration is increasingly turning the mobile network into a gigantic ubiquitous sensing platform, enabling large-scale analysis and applications. Recently, mobile data-based research reached important conclusions about various aspects of human mobility patterns. But how accurately do these conclusions reflect the reality? To evaluate the difference between reality and approximation methods, we study in this paper the error between real human trajectory and the one obtained through mobile phone data using different interpolation methods (linear, cubic, nearest interpolations) taking into consideration mobility parameters. Moreover, we evaluate the error between real and estimated load using the proposed interpolation methods. From extensive evaluations based on real cellular network activity data of the state of Massachusetts, we show that, with respect to human trajectories, the linear interpolation offers the best estimation for sedentary people while the cubic one for commuters. Another important experimental finding is that trajectory estimation methods show different error regimes whether used within or outside the “territory” of the user defined by the radius of gyration. Regarding the load estimation error, we show that by using linear and cubic interpolation methods, we can find the positions of the most crowded regions (“hotspots”) with a median error lower than 7%.

~\cite{zheng2010understanding}
User mobility has given rise to a variety of Web applications, in which the global positioning system (GPS) plays many important roles in bridging between these applications and end users. As a kind of human behavior, transportation modes, such as walking and driving, can provide pervasive computing systems with more contextual information and enrich a user’s mobility with informative knowledge. In this article, we report on an approach based on supervised learning to automatically infer users’ transportation modes, including driving, walking, taking a bus and riding a bike, from rawGPS logs. Our approach consists ofthree parts: a change point-based segmentation method, an inference model and a graph-based post-processing algorithm. First, we propose a change point-based segmentation method to partition each GPS trajectory into separate segments of different transportation modes. Second, from each segment, we identify a set of sophisticated features, which are not affected by differing traffic conditions (e.g., a person’s direction when in a car is constrained more by the road than any change in traffic conditions). Later, these features are fed to a generative inference model to classify the segments of different modes. Third, we conduct graph-based postprocessing to further improve the inference performance. This postprocessing algorithm considers both the commonsense constraints ofthe real world and typical user behaviors based on locations in a probabilistic manner. The advantages ofour method over the related works include three aspects. (1) Our approach can effectively segment trajectories containing multiple transportation modes. (2) Our work mined the location constraints from user-generated GPS logs, while being independent of additional sensor data and map information like road networks and bus stops. (3) The model learned from the dataset of some users can be applied to infer GPS data from others. Using the GPS logs collected by 65 people over a period of 10 months, we evaluated our approach via a set of experiments. As a result, based on the change-point-based segmentation method and Decision Tree-based inference model, we achieved prediction accuracy greater than 71 percent. Further, using the graph-based post-processing algorithm, the performance attained a 4-percent enhancement.

~\cite{widhalm2015discovering}
Massive and passive data such as cell phone traces provide samples of the whereabouts and movements of individuals. These are a potential source of information for models of daily activities in a city. The main challenge is that phone traces have low spatial precision and are sparsely sampled in time, which requires a precise set of techniques for mining hidden valuable information they contain. Here we propose a method to reveal activity patterns that emerge from cell phone data by analyzing relational signatures of activity time, duration, and land use. First, we present a method of how to detect stays and extract a robust set of geolocated time stamps that represent trip chains. Second, we show how to cluster activities by combining the detected trip chains with land use data. This is accomplished by modeling the dependencies between activity type, trip scheduling, and land use types via a Relational Markov Network. We apply the method to two different kinds of mobile phone datasets from the metropolitan areas of Vienna, Austria and Boston, USA. The former data includes information from mobility management signals, while the latter are usual Call Detail Records. The resulting trip sequence patterns and activity scheduling from both datasets agree well with their respective city surveys, and we show that the inferred activity clusters are stable across different days and both cities. This method to infer activity patterns from cell phone data allows us to use these as a novel and cheaper data source for activity-based modeling and travel behavior studies.

~\cite{biljecki2013transportation}
The knowledge of the transportation mode used by humans (e.g. bicycle, on foot, car and train) is critical for travel behaviour research, transport planning and traffic management. Nowadays, new technologies such as the Global Positioning System have replaced traditional survey methods (paper diaries, telephone) because they are more accurate and problems such as under reporting are avoided. However, although the movement data collected (timestamped positions in digital form) have generally high accuracy, they do not contain the transportation mode. We present in this article a new method for segmenting movement data into single-mode segments and for classifying them according to the transportation mode used. Our fully automatic method differs from previous attempts for five reasons: (1) it relies on fuzzy concepts found in expert systems, that is membership functions and certainty factors; (2) it uses OpenStreetMap data to help the segmentation and classification process; (3) we can distinguish between 10 transportation modes (including between tram, bus and car) and propose a hierarchy; (4) it handles data with signal shortages and noise, and other real-life situations; (5) in our implementation, there is a separation between the reasoning and the knowledge, so that users can easily modify the parameters used and add new transportation modes. We have implemented the method and tested it with a 17-million point data set collected in the Netherlands and elsewhere in Europe. The accuracy of the classification with the developed prototype, determined with the comparison of the classified results with the reference data derived from manual classification, is 91.6%.

~\cite{wang2010accelerometer}
Recognizing the transportation modes of people’s daily living is an important research issue in the pervasive computing. Prior research in this field mainly uses Global Positioning System (GPS), Global System for Mobile Communications (GSM) or their combination with accelerometer to recognize transportation modes, such as walking, driving, etc. In this paper, we will introduce transportation mode recognition on mobile phones only using embedded accelerometer. In order to deal with uncertainty of position and orientation of mobile phone, acceleration synthesization based method and acceleration decomposition based method are introduced. Performance comparison indicates that acceleration synthesization based method outperforms acceleration decomposition based method. We will discuss the factors affect the recognition accuracy of acceleration decomposition based method and present potential improvements.

~\cite{stenneth2011transportation}
The transportation mode such as walking, cycling or on a train denotes an important characteristic of the mobile user's context. In this paper, we propose an approach to inferring a user's mode of transportation based on the GPS sensor on her mobile device and knowledge of the underlying transportation network. The transportation network information considered includes real time bus locations, spatial rail and spatial bus stop information. We identify and derive the relevant features related to transportation network information to improve classification effectiveness. This approach can achieve over 93.5% accuracy for inferring various transportation modes including: car, bus, aboveground train, walking, bike, and stationary. Our approach improves the accuracy of detection by 17% in comparison with the GPS only approach, and 9% in comparison with GPS with GIS models. The proposed approach is the first to distinguish between motorized transportation modes such as bus, car and aboveground train with such high accuracy. Additionally, if a user is travelling by bus, we provide further information about which particular bus the user is riding. Five different inference models including Bayesian Net, Decision Tree, Random Forest, Naïve Bayesian and Multilayer Perceptron, are tested in the experiments. The final classification system is deployed and available to the public.

~\cite{trasarti2015discovering}
Mobile communication technologies pervade our society and existing wireless networks are able to sense the movement of people, generating large volumes of data related to human activities, such as mobile phone call records. At the present, this kind of data is collected and stored by telecom operators infrastructures mainly for billing reasons, yet it represents a major source of information in the study of human mobility. In this paper, we propose an analytical process aimed at extracting interconnections between different areas of the city that emerge from highly correlated temporal variations of population local densities. To accomplish this objective, we propose a process based on two analytical tools: (i) a method to estimate the presence of people in different geographical areas; and (ii) a method to extract time- and space-constrained sequential patterns capable to capture correlations among geographical areas in terms of significant co-variations of the estimated presence. The methods are presented and combined in order to deal with two real scenarios of different spatial scale: the Paris Region and the whole France.

~\cite{gambs2012next}
In this paper, we address the issue of predicting the next location of an individual based on the observations of his mobility behavior over some period of time and the recent locations that he has visited. This work has several potential applications such as the evaluation of geo-privacy mechanisms, the development of location-based services anticipating the next movement of a user and the design of location-aware proactive resource migration. In a nutshell, we extend a mobility model called Mobility Markov Chain (MMC) in order to incorporate the n previous visited locations and we develop a novel algorithm for next location prediction based on this mobility model that we coined as n-MMC. The evaluation of the efficiency of our algorithm on three different datasets demonstrates an accuracy for the prediction of the next location in the range of 70% to 95% as soon as n = 2.

~\cite{ficek2012inter}
With global mobile phone penetration nearing 100\%, cellular Call Data Records (CDRs) provide a large-scale and ubiquitous, but also sparse and skewed snapshot of human mobility. It may be difficult or inappropriate to reach strong conclusions about user movement based on such data without proper understanding of user movement between call records. Based on an analysis of a real-world trace, we propose a novel, probabilistic Inter-Call Mobility (ICM) model of users' position in between calls. The ICM model combines Gaussian mixtures to build a general, comprehensive spatio-temporal refinement of CDRs.We demonstrate that ICM model's application yields strikingly different conclusions to the existing models when applied to basic CDR analyses, such as user proximity probability.

~\cite{shin2015urban}
We present a prototype mobile phone application that implements a novel transportation mode detection algorithm. The application is designed to run in the background, and continuously collects data from built-in acceleration and network location sensors. The collected data is analyzed automatically and partitioned into activity segments. A key finding of our work is that walking activity can be robustly detected in the data stream, which, in turn, acts as a separator for partitioning the data stream into other activity segments. Each vehicle activity segment is then sub-classified according to the vehicle type. Our approach yields high accuracy despite the low sampling interval and does not require GPS data. As a result, device power consumption is effectively minimized. This is a very crucial point for large-scale real-world deployment. As part of an experiment, the application has been used by 495 samples, and our prototype provides 82\% accuracy in transportation mode classification for an experiment performed in Zurich, Switzerland. Incorporating location type information with this activity classification technology has the potential to impact many phenomena driven by human mobility and to enhance awareness of behavior, urban planning, and agent-based modeling.

~\cite{6181051}
Transportation activity surveys investigate when, where and how people travel in urban areas to provide information necessary for urban transportation planning. In Singapore, the Land Transport Authority (LTA) carries out such a survey amongst households every four years. The survey is conducted through conventional questionnaires and travel diaries. However, the conventional surveys are problematic and error-prone. We are developing a smartphone-based transportation activity survey system to replace the traditional household surveys, which can potentially be used by LTA in future.

~\cite{manzoni2010transportation}
In a context where personal mobility accounts for about two thirds of the total transportation energy use, assessing an individual’s personal contribution to the emissions of a city becomes highly valuable. Prior efforts in this direction have resulted in web-based CO2 emissions calculators, smartphonebased applications, and wearable sensors that detect a user’s transportation modes. Yet, high energy consumption and had-hoc sensors have limited the potential adoption of these methodologies. In this technical report we outline an approach that could make it possible to assess the individual carbon footprint of an unlimited number of people. Our application can be run on standard smartphones for long periods of time and can operate transparently. Given that we make use of an existing platform (smartphones) that is widely adopted, our method has the potential of unprecedented data collection of mobility patterns. Our method estimates in real-time the CO2 emissions using inertial information gathered from mobile phone sensors. In particular, an algorithm automatically classifies the user’s transportation mode into eight classes using a decision tree. The algorithm is trained on features computed from the Fast Fourier Transform (FFT) coefficients of the total acceleration measured by the mobile phone accelerometer. A working smartphone application for the Android platform has been developed and experimental data have been used to train and validate the proposed method.


~\cite{6038808}
The aim of this study is to develop a system for investigating human falls and mobility based on a Smartphone platform. We have designed and tested a set of software applications building on the inertial data captured from the tri-axial accelerometer sensor embedded in the Smartphone. We will describe here two applications: a fall detection and management application, and an application for the administration of a popular and standardized test in the field of human mobility assessment, namely the Timed-Up-and-Go test.

~\cite{5283030}
An important tool for evaluating the health of patients who suffer from mobility-affecting chronic diseases such as MS, Parkinson's, and muscular dystrophy is assessment of how much they walk. Ambulation is a mobility monitoring system that uses Android and Nokia N95 mobile phones to automatically detect the user's mobility mode. The user's only required interaction with the phone is turning it on and keeping it with him/her throughout the day, with the intention that it could be used as his/her everyday mobile phone for voice, data, and other applications, while Ambulation runs in the background. The phone uploads the collected mobility and location information to a server and a secure, intuitive Web-based visualization of the data is available to the user and any family, friends or caregivers whom they authorize, allowing them to identify trends in their mobility and measure progress over time and in response to varying treatments.

~\cite{6460199}We propose a novel method for automatic detection of the transport mode of a person carrying a Smart-phone. Existing approaches assume idealized positioning data with no GPS signal losses, require information from additional external sources such as real time bus locations, or only allow for a coarse distinction between very few categories (e.g. `still', `walk', `motorized'). Our approach is designed to deal with cluttered real-world Smartphone data and can distinguish between fine-grained transport mode categories. It is robust against GPS signal losses by including positioning data obtained from the cellular network and data from accelerometer readings. Mode detection is performed by a two-stage classification technique using randomized ensemble of classifiers combined with a Hidden Markov Model. We report promising results of an experimental performance analysis with real-world data collected by 15 volunteers during their everyday routines over a period of two months.

~\cite{sohn2006mobility}
Recognition of everyday physical activities is difficult due to the challenges of building informative, yet unobtrusive sensors. The most widely deployed and used mobile computing device today is the mobile phone, which presents an obvious candidate for recognizing activities. This paper explores how coarse-grained GSM data from mobile phones can be used to recognize high-level properties of user mobility, and daily step count. We demonstrate that even without knowledge of observed cell tower locations, we can recognize mobility modes that are useful for several application domains. Our mobility detection system was evaluated with GSM traces from the everyday lives of three data collectors over a period of one month, yielding an overall average accuracy of 85%, and a daily step count number that reasonably approximates the numbers determined by several commercial pedometers.

~\cite{Reddy:2010:UMP:1689239.1689243}
As mobile phones advance in functionality and capability, they are being used for more than just communication. Increasingly, these devices are being employed as instruments for introspection into habits and situations of individuals and communities. Many of the applications enabled by this new use of mobile phones rely on contextual information. The focus of this work is on one dimension of context, the transportation mode of an individual when outside. We create a convenient (no specific position and orientation setting) classification system that uses a mobile phone with a built-in GPS receiver and an accelerometer. The transportation modes identified include whether an individual is stationary, walking, running, biking, or in motorized transport. The overall classification system consists of a decision tree followed by a first-order discrete Hidden Markov Model and achieves an accuracy level of 93.6\% when tested on a dataset obtained from sixteen individuals.

~\cite{Hemminki:2013:ATM:2517351.2517367}
We present novel accelerometer-based techniques for accurate and fine-grained detection of transportation modes on smartphones. The primary contributions of our work are an improved algorithm for estimating the gravity component of accelerometer measurements, a novel set of accelerometer features that are able to capture key characteristics of vehicular movement patterns, and a hierarchical decomposition of the detection task. We evaluate our approach using over 150 hours of transportation data, which has been collected from 4 different countries and 16 individuals. Results of the evaluation demonstrate that our approach is able to improve transportation mode detection by over 20\% compared to current accelerometer-based systems, while at the same time improving generalization and robustness of the detection. The main performance improvements are obtained for motorised transportation modalities, which currently represent the main challenge for smartphone-based transportation mode detection.



~\cite{Alsolami2012Auth,jiang2013review} infer user trajectory from cell-phone traces based on how likely a specific route can lead to similar tower access sequences stored in the data traces. ~\cite{bekhor2015investigation} proposed an approach that can deal with common zig-zag problems in inferring user mobility from smartphone traces. ~\cite{doyle2011utilising} use similar method to distinguish long distance transportation modes such as cars and trains.

On the contrary, ~\cite{wang2010transportation} does not try to estimate a user's exact trajectory from smartphone traces, instead it classify a user's transportation mode by clustering on travel time distribution. 

\subsection{Geospatial app usage}
~\cite{shafiq2012characterizing,yang2015characterizing} studied the geospatial relation of app usage volume. Their works mostly studied the spatial correlation of smartphone usage and user mobility's impact on app usage is still a missing piece of these works.

~\cite{yan2012fast}
As mobile apps become more closely integrated into our everyday lives, mobile app interactions ought to be rapid and responsive. Unfortunately, even the basic primitive of launching a mobile app is sorrowfully sluggish: 20 seconds of delay is not uncommon even for very popular apps.

We have designed and built FALCON to remedy slow app launch. FALCON uses contexts such as user location and temporal access patterns to predict app launches before they occur. FALCON then provides systems support for effective app-specific prelaunching, which can dramatically reduce perceived delay.

FALCON uses novel features derived through extensive data analysis, and a novel cost-benefit learning algorithm that has strong predictive performance and low runtime overhead. Trace-based analysis shows that an average user saves around 6 seconds per app startup time with daily energy cost of no more than 2% battery life, and on average gets content that is only 3 minutes old at launch without needing to wait for content to update. FALCON is implemented as an OS modification to the Windows Phone OS.


~\cite{smoreda2013spatiotemporal}
In this paper, we will review several alternative methods of collecting data from mobile phones for human mobility analysis. We will briefly describe cellular phone network architecture and the location data it can provide, and will discuss two types of data collection: active and passive localization. Active localization is something like a personal travel diary. It provides a tool for recording positioning data on a survey sample over a long period of time. Passive localization, on the other hand, is based on phone network data which are automatically recorded for technical or billing purposes. It offers the advantage of access to very large user populations for mobility flow analysis of a broad area. We propose considering cellular network location data as a useful complementary source for human mobility research and provide case studies to illustrate the advantages and disadvantages of each method

~\cite{meng2014analyzing}
Over the past few years, mobile devices, particularly smartphones have seen dramatic increases in data consumption. The significant increases in data usage have placed tremendous strain on the wireless infrastructure necessitating research across a variety of optimization, efficiency, and capacity improvements. Complementary to those research efforts is the acquisition of a better understanding what aspects drive user smartphone usage. In this paper, we leverage the unique characteristics of the NetSense study to demonstrate how proximity, location, and individual differences (e.g., personality) can play an important role in understanding user behavior.

%Dissertation: ~\cite{Alsolami2012Auth} using only mobile handovers to infer client's location and speed information. The first naive way to do this is only for driving users. The tower powers which are used to calculate the boundary and physical roadways needs to know to determine the exact path of a particular user is driving. The problem is that the assumption that signal strength of tower can not be deterministically calculated at a certain point is an invalid assumption. 
%Difference of our case: They only calculate traces, not speed. 
%
%Similar work: ~\cite{jiang2013review} review ubiquitous findings in human mobility and present current computational challenges involved in treating data for inferring trip purpose and road usage. The first feature is the presence of preferential returns to visited locations mixed with the exploration of new ones. The second feature is the extraction of daily mobility motifs. Two major challenges of using mobile phone data is that, first there are indefinite gaps in space and time. Second, the accuracy is low.
%Our data is more frequent. Thus can better determine mobility mode, rather than trace only
%
%Another way to infer user's mobility mode is from the CDR (call detail records) which is introduced in ~\cite{wang2010transportation}. The challenge here is to inferring transportation mode based on coarse-grained CDRs. The algorithm can achieve acceptable accuracy with very low cost and complexity.

