With the rapid growth in smartphone usage, it has been more and more important to understand the patterns of mobile data consumption by users.
In this paper, we present an empirical study 
of the correlation between user mobility and app usage patterns.
In particular, we focus on users' moving speed as the key mobility metric,
and try to answer the following question:
are there any notable relations between moving speed and the app usage patterns?
Our study is based on a real-world, large-scale dataset of 2G phone network data request records.
A critical challenge was that the raw data records are rather coarse-grained.
More specifically, unlike GPS traces, the exact locations of users were not readily available.
We inferred users' approximate locations according to their interactions with nearby cell towers, whose locations were known.
We %address the challenge of user speed estimation by 
proposed a novel method to filter out noises and perform reliable speed estimation.
We verify our methodology with out of sample data and show its improvement in speed estimation accuracy.
We then examined several aspects of mobile data usage patterns,
including the data volume, the access frequency, and the app categories,
to reveal the correlation between these patterns and users' moving speed.
Experimental results based on our large-scale real-world datasets revealed that users under different mobility categories not only have different smartphone usage motivations but also have different ways of using their smartphones.