\newpage
\begin{verbatim}
>> RESPONSE TO REVIEWER 2 <<
 
>  ## 1. Motivation for how the results of this work can be used and their
>  generalization.
>
>  The authors present correlations between user movement speed and other
>  features such as number of applications used, traffic volume, idle time,
>  etc.
>  The authors motivate such findings with: “Understanding such
>  correlations, if any, could provide useful contextual information for
>  relevant and accurate app recommendation and ad delivery. For example, if
>  we find out hiking hobbyists use certain apps considerably more often,
>  then such apps may be more useful venues for ad delivery for equipment
>  makers for hiking activities.”
>
>  However, from the findings presented in this work I only see that Fig16
>  supports the aforementioned motivation by providing insights about the
>  correlation between movement speed and app categories.
>  Providing stronger / additional arguments why all the other observations
>  are relevant would help the authors to increase the value of this work.
\end{verbatim}

We appreciate your thoughtful suggestions. 

\change{We mark the related change in paper as R2-1.}

\begin{verbatim}
>  Furthermore, as mentioned in the paper (however, only once I guess), the
>  traces used for this work cover cellular communication only.
>  As the authors have pointed out, part of the communication might be
>  handled over Wi-Fi.
>  In recent studies, authors have observed that over 60% of the time users
>  are connected to Wi-Fi [1] and that half of the traffic is typically
>  handled over Wi-Fi [2].
>
>  So the important question at this point is: how representative are the
>  results presented in this work given the fact that they consider cellular
>  app usage and traffic only?
\end{verbatim}

Thank you for your comments. 
Please see response to meta-review comment 4.

\begin{verbatim}
> ## 2. My second concern is about the performance in estimating movement
>  speed that has a direct influence on the results presented in this work.
>
>  To estimate intra-cell speed, the authors propose a novel algorithm that
>  is based on a set of assumptions such as that users move with a constant
>  speed (4.1) or with a “straight line trajectory” (4.3).
>  The resulting computation of the movement speed is therefore a direct
>  consequence of the aforementioned assumptions and the introduced
>  approach.
>  Therefore, the results presented in this work rely on the quality of this
>  computation.
>
>  The results presented in this work show a positive correlation (Fig10a)
>  between movement speed and traffic volume / sec. as well as a negative
>  correlation (Fig10b) between the idle time and speed.
>  The conclusion is that the faster a user passes a given cell, the more
>  bytes/s she will generate and the smaller the idle time intervals are.
>  So to make sure that these insights indeed cover a valid collection
>  between movement speed and other features, it is mandatory to show that
>  the aforementioned assumptions hold, in general, and the movement speed
>  estimation is accurate, to some extent, allowing to make conclusions from
>  the experiments.
>
>  Addressing this shortcoming might for instance include an evaluation of
>  the approach on a data set that contains GPS and cellular data.
>  There are several publicly available data sets that might be helpful at
>  this stage (reality mining, nokia, lifemap, etc.).
>  Alternatively, running a small custom study to verify the assumptions and
>  get quantitative evidence that movement speed estimations are accurate,
>  might also be an option.
>
>  Without showing that the novel approach presented in this work to compute
>  movement speed produces reliable estimates, it is at least possible that
>  to some extent the insights presented in this work result from the
>  inaccurate movement speed estimations.
\end{verbatim}

We are really appreciate for your constructive comments.
Please see response to meta-review comment 3.

\begin{verbatim}
> ## Minor comments:
>  - Please try to put Figs and Listings on the same page as the text
>  which refers to them
>  - Inconsistent writing of PBE in Sec 4.1 and Sec 4.2
>  - Potential naming (variables) inconsistencies between Alg1 and Sec 4.2
>  - Fig10a y-axis label: is it not supposed to be “bytes/sec” as
>  explained in the corresponding section?
>  - Sec 5.3: how do you define “a data access”. is it a single CDR in
>  the data set?
>  - What is the overall value / take-home message of Fig11 and Fig12? Is
>  it not better to have relative values if the number of instances differ?
>  - Fig14 and Fig15 should be a bit smaller to match the font size of the
>  text
\end{verbatim}

Thank you for identifying these problems. 
We corrected these problems accordingly. 
We are sorry that we are not able to use red color to highlight some changes.

\begin{verbatim}
>  Suggested References:
>  [1] Kyunghan Lee,Joohyun Lee,Yung Yi,Injong Rhee, and Song Chong. 2013.
>  Mobile Data Offloading: How Much can WiFi Deliver? IEEE/ACM Transactions
>  On Networking 21, 2 (2013), 536551.
>
>  [2] Paul Baumann and Silvia Santini. 2014. How the availability of Wi-Fi
>  connections influences the use of mobile devices. In Proceedings of the
>  2014 ACM International Joint Conference on Pervasive and Ubiquitous
>  Computing Adjunct Publication - UbiComp ’14 Adjunct. (2014), 367372.
\end{verbatim}

Please see response to meta-review comment 5.
