\newpage
\begin{verbatim}
>> RESPONSE TO REVIEWER 4 <<

>  the use case presented (that of targeted advertising) needs evidencing
>  or reframing. There are many ways to target user interests and it is not
>  obvious that an approach based on speed is appropriate. If there is no
>  evidence to support the use of speed in targeted adverts then I would
>  have preferred to have seen the approach based as a generic analysis with
>  a range of possible applications being suggested.
\end{verbatim}

We appreciate your thoughtful suggestions. 

\change{We mark the related change in paper as R4-1.}

\begin{verbatim}
>  clearly the data captured is only part of a user's overall data
>  consumption. This point is made in the paper but should be made more
>  explicitly at the start.
\end{verbatim}

Please see response to meta-review comment 1.

\begin{verbatim}
>  the length of the data trace (3 hours) seems short - it would be good
>  to explain why the authors feel this is an appropriate amount of data to
>  analyse.
\end{verbatim}

Thank you for your comments. Unfortunately, this is all the data that available to us. We have discussed this limitation of our dataset in the new limitation section. Eventhough our dataset is relatively short, the speed estimation method can be easily applied to similar datasets of any given time period. Similar procedure can be used to study usage patterns in those datasets without much difficulty.

\change{We mark the related change in paper as M2b.}

\begin{verbatim}
>  the speed estimation approach contains a very large number of
>  assumptions and there is no evidence presented that it actually works.
>  This is the biggest weakness in the paper - it *must* provide some
>  evidence of success. Even a simple study with 20 users where ground truth
>  and cell records were collected would be sufficient. Without that there
>  is simply not enough to convince the reader that the algorithm presented
>  works.
\end{verbatim}

Please see response to meta-review comment 3.

\begin{verbatim}
>  the analysis would be much stronger if it was backed up with some
>  evidence (e.g. a survey or a focus group or some observations studies)
>  that trued to explain the patterns seen.
\end{verbatim}

Thank you for your comments.
However, we are not able to find previous survey or observation studies on this topic.
