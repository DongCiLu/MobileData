\noindent Dear Editors and Reviewers,
\bigskip

Thank you very much for your insightful feedback and constructive suggestions. We have thoroughly addressed your comments %to our best effort 
in this revision. %, and list them as follows, along with our revised manuscript. 
In the following, we will first summarize main changes made to the manuscript, and then  provide a point-to-point response to comments from both the meta-review and each reviewer.

For each point of change, we assigned a symbol that corresponds to a specific review comment. For example, M3 means Comment 3 in the meta-review, and R1-3 means Comment 3 from Reviewer 1, and so on.
Changes to the manuscript are highlighted in blue.

\bigskip
In this revision, we primarily made the following changes. 

\begin{enumerate}

\item (M3) %Based on the comment from the meta-review, 
We performed additional evaluations of our speed estimation algorithm using a dataset that contains ground-truth information (see Section 5 on Page 12).
%The dataset contains both cell phone data and GPS data. This way we were able to evaluate the accuracy of our proposed algorithm and demonstrate its effectiveness by using the GPS data as the ground truth. 

\item We added a section to discuss the limitations of our work (see Section 7). We highlighted when our system would be useful and valid, and when the results would require additional assumptions. More specifically, 
	\begin{itemize}
    \item (M4) We discussed how to integrate WiFi data into our analysis when such data are available; %, as our current study does not involve Wifi data. In particular, we describe the preliminary steps for integration for WiFi data.  Changes can be found on 
    %See Section 7 on Page 17.  
    \item (M2b, R1-3) We discussed possible limitations of our data with respect to the coverage of time and % length of our dataset and the limited 
    geographic areas; %, and their potential impact on the results. As this limitation is introduced because of the nature of the dataset, we cannot easily overcome them without new data. However, we indeed used open-source data to carry out additional experiments as in the previous change. 
    %Changes can be found on page 18 in section 7.
	\item (M2c) We discussed the limitations introduced by coarse-grained location information and irregular time intervals in the dataset. % in Section 7 on page 18. 
\end{itemize}

\item We added more information about our study as suggested by the reviewers.
%More specifically, 
	\begin{itemize}
	\item (M1) We provided more background information and description about our dataset (see Section 3 on Page 4);
  	\item (M5) We added and discussed additional references (see Section 2 on Page 4);
  	\item (R1-4) We added more details about implementation and environment settings (see Section 6 on Page 13).
\end{itemize}

\item (R2-1, R4-1)  We revised the motivation of our study to highlight the need and usefulness of our approach (see Section 1 on Page 2). 
\item We re-arranged the figures to make them close to the corresponding text, and also fixed several naming issues. 
  %\change{remaining change: figure and listing on the same page as text - done}
  \item We proofread the paper thoroughly to make sure that there is no more language or formatting issue. 
\end{enumerate}

