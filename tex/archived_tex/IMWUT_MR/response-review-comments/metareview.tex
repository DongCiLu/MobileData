\newpage
\begin{verbatim}
>> RESPONSE TO META-REVIEW <<

>  1. (R1) Describe how the data was collected. Who were involved? How long
>  the data was collected? When the data was collected? Is it cell data the
>  only collected data or was WiFi also collected?
\end{verbatim}

(M1) Thank you for this comment. Our dataset was collected by a major mobile carrier in China as the operator for cell phone towers. All active GSM data access traces during a three-hour period from 6pm to 9pm on a Sunday in September 2014 were recorded. We added this background information to the paper, and marked the related changes in the paper as (M1). See Section 3 on Page 4.

This dataset only contained cellular data and did not contain any WiFi data. Despite this limitation, we emphasize that the large data size allows us to carry out a study on the real-world app usage with a larger scale and better accuracy than related work. 

\begin{verbatim}
> 2. (R1 and R2) The authors should address the generality of the proposed
> algorithm as a limitation. Generally, smartphone users use both cellular
> and wifi networks in other areas. The current analysis model excludes
> many contexts of users such as location, time, individual context, etc.
> In order to understand the resulting data, it is very important to know
> about moving speed with respect to users and their culture.
\end{verbatim}

Thank you for pointing out the generality issue. In this revision, we added a separate section to discuss these issues and possible limitations. 

a. (M2a) On the limitation of using WiFi data:
%Thank you for this comment. 
We recognize that although WiFi handles half of the mobile data traffic and more than half of time users are connected to WiFi, cell phone networks typically have much better coverage and user mobility diversity. It is more meaningful to study user mobility under cellular traffic from the operators' perspective to understand large-scale user preferences and behavioral patterns. We do acknowledge that the lack of WiFi data is a limitation of our work and that our current speed estimation algorithm cannot be directly applied to WiFi data. However, it is very hard to find WiFi data on a whole city scale, and to our knowledge, no such large-scale dataset has yet collected. Therefore, we highlight this as a limitation due to such practical concerns. 
We mark the related changes in the paper in Section 7 on Page 17.


b. (M2b) Time and Location contexts:
Due to the short time period of our dataset, it was not feasible for us to perform a thorough temporal %or spatial 
analysis for much longer periods of time, such as days or weeks. 
Nevertheless, the speed estimation method can be easily applied to similar datasets from other areas of any given time period to explore time and location contexts. 
For example, in our evaluations section, we used a different dataset in the U.S. to verify our speed estimation algorithm. We discussed this issue in the limitations section. 
We mark the related change in the paper in Section 7 on Page 18.

c. (M2c) Individual contexts:
We agree that our results only represent those users who frequently moved and use mobile network. Because the dataset is fundamentally coarse-grained, we can only analyze limited location information and irregular time intervals. We state this limitation in our new limitation section in Section 7 on Page 18.  

\begin{verbatim}
>  3. (R2) Speed estimation algorithm is not proved yet. The authors should
>  demonstrate that their approach to estimate intra-cell movement speed is
>  accurate and therefore can be reliable used for the correlation analysis.
>  Many assumptions in estimating moving speed are made without much
>  explanation. These assumptions should be proved and discussed in detail.
>  In addition, the classes of moving speed should be analyzed and discussed
>  in detail.
\end{verbatim}

%Since it is not feasible for us to collect ground truth of user mobility in our dataset, 
(M3) Following one of the reviewer's suggestions, we used a dataset collected by Intel Placelab %similar but much smaller 
to perform additional evaluations of our speed estimation algorithm. This dataset contains both cell phone data and GPS traces. We used the GPS data as the ground truth for  evaluating our speed estimation algorithm that utilizes cell phone data only. We also added more details regarding the assumptions on estimating the moving speeds in our algorithm design. The moving speed classes are also analyzed and discussed with more details. 

%Even though the dataset was collected from a different country (US), our speed estimation algorithm worked well with same set of parameters as those used in our experiment section. We add a new evaluation section to show the results. 
The changes are available in Section 5 on Page 12. 
 

\begin{verbatim}
> 4. (R1 and R2) The authors should discuss or demonstrate the implications
> of considering cellular traffic only on the results presented in this
> work. WiFi usage could be a major limitation of the generalization of the
> study.
\end{verbatim}

Please see our response 
%Our response to WiFi related comments is 
listed under meta-review Comment 2a (M2a) above. 
%We mark the related change in paper on page 17 in section 7.

\begin{verbatim}
>  5. Suggested References:
>  [1] Kyunghan Lee,Joohyun Lee,Yung Yi,Injong Rhee, and Song Chong. 2013.
>  Mobile Data Offloading: How Much can WiFi Deliver? IEEE/ACM Transactions
>  On Networking 21, 2 (2013), 536-551.
>
>  [2] Paul Baumann and Silvia Santini. 2014. How the availability of Wi-Fi
>  connections influences the use of mobile devices. In Proceedings of the
>  2014 ACM International Joint Conference on Pervasive and Ubiquitous
>  Computing Adjunct Publication - UbiComp '14 Adjunct. (2014), 367-372.
\end{verbatim}

Thank you. 
We have added all suggested references and discussed them in the related work section. 
These changes can be found in Section 2 on Page 4 of the paper.