\newpage
\begin{verbatim}
>> RESPONSE TO REVIEWER 4 <<

>  the use case presented (that of targeted advertising) needs evidencing
>  or reframing. There are many ways to target user interests and it is not
>  obvious that an approach based on speed is appropriate. If there is no
>  evidence to support the use of speed in targeted adverts then I would
>  have preferred to have seen the approach based as a generic analysis with
>  a range of possible applications being suggested.
\end{verbatim}

(R4-1) We appreciate the insightful suggestions. We agree that there are many ways to target user interests. However, our primary goals in this paper aim to study whether speed differences of users will have a significant impact on their usage patterns in a large scale population. Although other factors may also have an impact on usage patterns, we consider that they are out of the scope of this paper. We have added additional arguments on the observations and their support for the conclusions in the paper. The changes can be found on page 2 in section 1. 

\begin{verbatim}
>  clearly the data captured is only part of a user's overall data
>  consumption. This point is made in the paper but should be made more
>  explicitly at the start.
\end{verbatim}

Please see response to meta-review comment 1.

\begin{verbatim}
>  the length of the data trace (3 hours) seems short - it would be good
>  to explain why the authors feel this is an appropriate amount of data to
>  analyse.
\end{verbatim}

(R4-3) Thank you for your comments. As we mentioned in the summary of responses, this is all the data that available to us. We have discussed this limitation of our dataset in the new limitation section. Even though our dataset is relatively short, the speed estimation method can be easily applied to similar datasets of any given time period. Similar procedures can be used to study usage patterns in those datasets without much difficulty. We have made changes to the paper to include this limitation in section 7 on page 17. 
 
\begin{verbatim}
>  the speed estimation approach contains a very large number of
>  assumptions and there is no evidence presented that it actually works.
>  This is the biggest weakness in the paper - it *must* provide some
>  evidence of success. Even a simple study with 20 users where ground truth
>  and cell records were collected would be sufficient. Without that there
>  is simply not enough to convince the reader that the algorithm presented
>  works.
\end{verbatim}

Please see our response to meta-review comment 3.

\begin{verbatim}
>  the analysis would be much stronger if it was backed up with some
>  evidence (e.g. a survey or a focus group or some observations studies)
>  that trued to explain the patterns seen.
\end{verbatim}

Thank you for your comments. 
Actually, in a recent work [1] which also study correlation of user mobility and mobile data access, similar trends as shown in Figure 9b have been found. 
Another work on geospatial relation of the app usage [2] showed similar smartphone app usage distribution as shown in Table 1 and Figure 11.  

Related references:

[1] Jie Yang, Yuanyuan Qiao, Xinyu Zhang, Haiyang He, Fang Liu, and Gang Cheng. 2015. Characterizing User Behavior in Mobile Internet.
IEEE Transactions on Emerging Topics in Computing 3, 1 (2015), 95--106.

[2] M Zubair Shafiq, Lusheng Ji, Alex X Liu, Jeffrey Pang, and Jia Wang. 2012. Characterizing geospatial dynamics of application usage in a
3G cellular data network. In INFOCOM. 1341--1349.

