\section{Discussions and Limitations}
\label{limitation}

Despite the results based on our analysis of the large-scale dataset, we acknowledge that our study still has a number of limitations. It is pivotal to discuss them so that our results can be interpreted in a meaningful manner.

%\change{We discuss three limitations of our work in this section. The first one is that our speed estimation method does not consider WiFi traffic. The second one is that our dataset only have limited time length and location area. The last one is that our speed estimation algorithm filtered out users that do not frequently move nor frequently use cell phone data networks.}

%\change{\subsection{(M4) WiFi data for speed estimation}}

First, our speed estimation algorithm is solely based on the coarse-grained location information from cell phone data access traces. We have not integrated more fine-grained WiFi data for the reason that cell phone network still has far better coverage compared to WiFi networks. Moreover, users on cell phone networks usually have a higher degree of speed variations, whereas users using WiFi are more likely to be stationary.

On the other hand, we discuss briefly on how to integrate WiFi data to improve speed estimation, if such data were available. We observe that the coverage patterns of WiFi access points (APs) are very different from cell phone towers. They are usually heavily overlapped with each other. Furthermore, WiFi APs usually have smaller coverages than cell phone towers. Therefore, if we have WiFi traces, we can estimate the locations of users more accurately by using triangularization methods. Such locations can serve as calibration records for the estimated trajectories. For example, when estimating the intra-cell boundary-to-boundary distances, instead of using a straight line trajectory that passes all adjacent tower as an estimated trajectory, we can use the trajectory that passes all recorded WiFi coverage areas in between as the better estimates. However, in this case, the distance lower bound should also be the shortest distance between two boundaries that passes all recorded WiFi coverage areas in between. Note that the distance lower bound might now be a straight line in this case.

%\change{\subsection{(M2b, R1-3) Limitation of dataset}}

A second limitation is that our dataset has limited temporal and spatial coverage. Therefore, we are not able to perform in-depth studies on temporal or spatial trends. Further, although our dataset contains data from three cities, they only represent patterns of densely populated areas. We believe that our speed estimation methods can be easily applied to similar datasets from other areas. For example, in our evaluation section, we use a smaller dataset from a whole different area to verify the performance of our speed estimation algorithm.

%\change{\subsection{(M2c) Limitation of speed estimation}}

Finally, our speed estimation algorithm utilizes passing-boundary events as building blocks. This technique requires that users have visited at least three nearby towers. This is because that, with tower coordinates as the only location information, it is almost impossible to infer user locations in a specific cell phone tower coverage area based on this tower alone. Therefore, if a user's trace is recorded by fewer than three towers, the uncertainty of speed estimation will undoubtedly increase.
