\noindent Dear Editors and Reviewers,
\bigskip

Thank you very much for your insightful feedback and constructive suggestions. We have thoroughly addressed your comments %to our best effort 
in this revision. %, and list them as follows, along with our revised manuscript. 
In the following, we provide a point-to-point response to comments from each reviewer.

\bigskip
\begin{verbatim}
>> RESPONSE TO REVIEWER 1 <<

>  There are couple of minor things that should be addressed. I think that
>  the authors describe high level tendency of app usage with respect to
>  speed. I'm not sure why the authors compared number of unique app and
>  unique app category such as Fig. 10 (a) and (d), (b) and (e), and (c)
>  and(f). Also I wonder there is statistically meaningful relationship in
>  the results? It is general that users have more time to use apps when
>  there are sitting or waiting while they are busy when they are driving or
>  moving. I think that the authors should improve the analysis on app usage
>  and its correlation to mobility in detail. The results from the app usage
>  and its relation to mobility should be discussed. In addition, it is also
>  interesting to see browsers which took 50 percent of the collected data.
>  However, there was no mentation about how browsers.
\end{verbatim}

\note{Pending change.}

\bigskip
\begin{verbatim}
>> RESPONSE TO REVIEWER 2 <<

>  It would be good to have a similar plot to Fig.7 that
>  shows the absolute error as a CDF and not just the fraction "e".
\end{verbatim}

\note{Pending change.}

\bigskip
We also added the following reference:

\note{Pending change.}   
[1]  Matthias Böhmer, Brent Hecht, Johannes Schöning, Antonio Krüger, and Gernot Bauer. 2011. Falling asleep with Angry Birds, Facebook and Kindle: a large scale study on mobile application usage. In Proceedings ofthe 13th international conference on Human computer interaction with mobile devices and services. ACM, 47-56
