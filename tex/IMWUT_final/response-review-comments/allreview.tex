\noindent Dear Editors and Reviewers,
\bigskip

Thank you very much for your insightful feedback and constructive suggestions. We have thoroughly addressed your comments %to our best effort 
in this revision. %, and list them as follows, along with our revised manuscript. 
We also proofread the paper thoroughly to make sure that there is no more language or formatting issue. 
In the following, we provide a point-to-point response to comments from each reviewer.

\bigskip
\begin{verbatim}
>> RESPONSE TO REVIEWER 1 <<

>  The authors describe high level tendency of app usage with respect to
>  speed. I'm not sure why the authors compared number of unique app and
>  unique app category such as Fig. 10 (a) and (d), (b) and (e), and (c)
>  and(f). Also I wonder there is statistically meaningful relationship in
>  the results? It is general that users have more time to use apps when
>  there are sitting or waiting while they are busy when they are driving or
>  moving. I think that the authors should improve the analysis on app usage
>  and its correlation to mobility in detail. The results from the app usage
>  and its relation to mobility should be discussed. 
\end{verbatim}

Changes corresponding to this comment are presented in the first paragraph of Section 6.3 on Page 16. 
Note that Figure 10 and Figure 11 have been renumbered as Figure 11 and Figure 12, respectively.

In this revision, we added a paragraph to discuss motivations and results shown in Figure 11 and Figure 12. 
In Figure 11, we showed the correlation between estimated user speed and app switching behavior, 
which may help to answer questions like ``do users concentrate on a few apps or frequently switch between many apps at different speed levels?'' 
In Figure 12, we showed changes in the network access of each app category, 
which may help to answer questions like ``do the users use the same type of apps when they are moving compared to when they are sitting still?''

\begin{verbatim}
>  In addition, it is also interesting to see browsers which took 50 percent 
>  of the collected data. However, there was no mentation about how browsers.
\end{verbatim}

Changes correaponding to this comment are made in Table 1 on Page 17 
and the second paragraph of Section 6.3 on page 16.

In particular, we added two new columns in Table 1 that show example apps for each category, 
and a short remark to explain the most common usage of that category when necessary. 
We also included an overall description of the table.

\begin{verbatim}
> Please refer following paper:
> Matthias Bohmer, Brent Hecht, Johannes Schoning, Antonio Kruger, and
> Gernot Bauer. 2011. Falling asleep with Angry Birds, Facebook and Kindle:
> a large scale study on mobile application usage. In Proceedings of the
> 13th International Conference on Human Computer Interaction with Mobile
> Devices and Services (MobileHCI '11). ACM, New York, NY, USA, 47-56. DOI:
> https://doi.org/10.1145/2037373.2037383
\end{verbatim}

We added the following reference and briefly discussed it in the first paragraph of related work section:

[4]  Matthias B\"{o}hmer, Brent Hecht, Johannes Sch\"{o}ning, Antonio Kr\"{u}ger, and Gernot Bauer. 2011. Falling asleep with Angry Birds, Facebook and Kindle: a large scale study on mobile application usage. In Proceedings ofthe 13th international conference on Human computer interaction with mobile devices and services. ACM, 47-56

\bigskip
\begin{verbatim}
>> RESPONSE TO REVIEWER 2 <<

>  It would be good to have a similar plot to Fig.7 that
>  shows the absolute error as a CDF and not just the fraction "e".
\end{verbatim}

Changes corresponding to this comment are presented in the first two paragraphs and Figure 7 of Section 5.2 on Page 13.

We added Figure 7(a) that shows the empirical CDF of the absolute error. We also added the definition of the absolute error and the discussion of the CDF of absolute error which shows the similar trend as relative error.

