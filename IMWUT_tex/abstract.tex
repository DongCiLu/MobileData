With the rapid growth of smartphone usage, it has been more and more important to understand the patterns of mobile data consumption by users.
In this paper, we present an empirical study 
of the correlation between user mobility and app usage patterns.
In particular, we focus on users' moving speed as the key mobility metric,
and try to answer the following question:
are there any notable relations between moving speed and the app usage patterns?
Our study is based on a real-world, large-scale dataset of 2G phone network data request records.
A critical challenge is that the raw data records are rather coarse-grained.
More specifically, unlike GPS traces, the exact locations of users are not accurately available.
We might infer the approximate location of a user according to his or her interactions with the cell towers, whose locations are known.
We address the challenge of user speed estimation by proposing a novel methodology to filter out noises,
so that we can achieve reliable and fine-grained speed estimation.
We then examine several aspects of mobile data usage patterns,
including the data volume, the access frequency, and the app categories,
to reveal the correlation between these patterns with users' moving speed.